\usepackage[bitstream-charter, uppercase=upright]{mathdesign} % Fonte

% METADADOS
\hypersetup{
            pdfauthor={Matheus Jonatha},
            pdftitle={Card de memorização de inglês},
            pdfsubject={Frases em inglês},
            pdfkeywords={mathjonatha,mthsjonatha,inglês,matheus,jonatha,estudo},
            pdfproducer={Produzido e gerado no Overleaf, compilado em XeLaTeX},
            pdfcreator={XeLaTeX}}
            
% Comandos criados
\newcommand{\titulo}[2]{
\begin{flushleft}
{\bf \fontsize{80}{48} \selectfont #1} \\ \vspace{3mm}
{\fontsize{40}{48} \selectfont #2}
\end{flushleft}
}


%%------------------------------------------
%% CABEÇALHO E RODAPÉ
%%------------------------------------------
% \definecolor{corlinha}{RGB}{0,1,42}

% \usepackage{fancyhdr}
% \usepackage{lastpage} % Pacote para pegar a ultima pagina na numeração
% \pagestyle{fancy}
% \renewcommand{\headrulewidth}{0pt} %linha horizontal no topo da pagina
%\renewcommand{\footrulewidth}{0.4pt} %linha horizontal no pé da pagina

% \fancyhead[L]{}
% \fancyhead[R]{
% \textbf{Monitoria AnnWay - Curso de Matemática Básica}
    % \begin{tikzpicture}[overlay]
    %     \node[draw=corlinha,
    %     circle,minimum width=.7cm, minimum height=.7cm,
    %     anchor=south west,
    %     fill=corlinha,font=\fontsize{11}{15}\sffamily,inner sep=1pt,outer sep=1pt]
    %     at (0,0){\textcolor{white}{\thepage}};
    % \end{tikzpicture}
% }
% \fancyhead[C]{}

% \fancyfoot[R]{
%     \begin{tikzpicture}[overlay]
%         \node[draw=corlinha,
%         circle,minimum width=.7cm, minimum height=.7cm,
%         anchor=south west,
%         fill=corlinha,font=\fontsize{10}{15}\sffamily,inner sep=1pt,outer sep=1pt]
%         at (0,0){\textcolor{white}{\bfseries \thepage/\pageref{LastPage}}};
%     \end{tikzpicture}
% }
% \fancyfoot[C]{}

%\setlength\parindent{0pt}
%\setlength\parskip{1.5ex}
%\setlength\parsep{1.5\parskip}
%\thispagestyle{empty}%\bigskip %Rodapé na primeira pagina



% Plano de fundo
% \usepackage{eso-pic,graphicx} % Para funcionar o background

% \AddToShipoutPictureBG{\includegraphics[width=\paperwidth,height=\paperheight]{wallpaper.pdf}}

%--------------------------------------------------------
% Configurações do cards
%--------------------------------------------------------
\definecolor{preto}{RGB}{0,0,0}
\definecolor{branco}{RGB}{255,255,255}

\pagesetup{3}{8}

% redefining default text style for front side
\renewcommand{\cardtextstylef}{\bfseries}
\renewcommand{\cardtextstyleb}{}

\renewcommand{\frfoot}{ \footnotesize \thecardno} % \footnotesize
\renewcommand{\brfoot}{ \footnotesize \thecardno}
% \renewcommand{\fchead}{\thecardno}

% redefining center head on frontside
\renewcommand{\fchead}{%
\vskip-5pt\fboxsep=0pt%
\colorbox{preto}{%
\small \textcolor{branco}{%
% \parbox{\cardwidth}{\vskip3pt \centering \bf expressões idiomáticas \vskip3pt}%
\parbox{\cardwidth}{\vskip3pt \centering \bf frases em inglês \vskip3pt}%
}%
}%
}

% \renewcommand{\bcfoot}{%
% \vskip-5pt\fboxsep=0pt%
% \colorbox{preto}{%
% \small \textcolor{branco}{%
% \parbox{\cardwidth}{\vskip3pt \centering english - translate \vskip3pt}
% }%
% }%
% }


\newcommand{\pronuncia}{ \\ \vspace{.3cm} \it \footnotesize}
